\documentclass{report}

\usepackage[utf8]{inputenc}
\usepackage[T1]{fontenc}
\usepackage[francais]{babel}

\usepackage{amsmath}
\usepackage{amssymb}
\usepackage{amsthm}

\usepackage[svgnames]{xcolor}
\usepackage{array}

\usepackage{tikz}
\usetikzlibrary{patterns}
\usepackage{caption}
\usepackage{subcaption}

\newcommand{\bb}{\underline{b}(i,t_1,t_2)}
\newcommand{\wb}{\underline{w}(i,t_1,t_2)}
\newcommand{\bmin}{b_i^{min}}
\newcommand{\bmax}{b_i^{max}}
\newcommand{\inter}[2]{{[}#1,#2{]}}


\newtheorem{Th}{Théorème}
\newtheorem{Lem}{Lemme}

%TODO FLPM : preuve bi(t) cste par morceaux (c'est mieux que rien)
% contre exemple pour changer seulement fti et ri sur un super papier quelque 
%part sur ton bureau


\begin{document}

\section{Fonction Linéaire}

Nous allons montrer que dans le cas où toutes les fonctions $f_i$ sont linéaires, 
il existe un ordonancement réalisable où toutes les fonctions d'allocations de 
ressources $b_i(t)$ sont constantes par morceaux. C'est l'énoncé du théorème 
suivant.

\begin{Th}
	Soit ${\cal I}$ une instance réalisable de CECSP. Il existe une solution dans 
	laquelle $\forall i\in A$, il existe une suite $(t_q)_{1\le q \le Q}$ telle 
	que $\forall t \in {[}t_q,t_{q+1}{[}, b_i(t)$ est constante.\\
\end{Th}

\begin{proof}
	Nous commencons par prouver la première partie du théorème.\\
	Pour cela, considérons une solution réalisable $S$ de ${\cal I}$ et soit 
	$(t_q)_{q=1,\dots ,Q}$ la suite des $st_i$ et $ft_i$, triés par ordre 
	croissant.\\
	Comme $S$ est une solution réalisable, on a: 
	\[\forall t \in {\cal T},\ \sum_{i\in A} b_i(t) \le B\]
	\[ \Rightarrow \forall q \in \{1,\dots , Q-1\},\  
	\int_{t_q}^{t_{q+1}} \sum_{i\in A}b_i(t) dt \le B(t_{q+1} - t_q)\]
	\[ \Rightarrow \sum_{i \in A} \frac{\int_{t_q}^{t_{q+1}} b_i(t)dt}
	{t_{q+1}-t_q} \le B\]
	Donc, exécuter $i$ à $\frac{\int_{t_q}^{t_{q+1}} b_i(t)dt}
	{t_{q+1}-t_q} $ ne viole pas la contrainte de ressource et produit la même
	 énergie. En effet,\\
	$
	\begin{array}{rl}
		\int_{t_q}^{t_{q+1}} f_i \left(\frac{\int_{t_q}^{t_{q+1}} b_i(t) dt}
		{t_{q+1} -t_q}\right) dt & 
		= \int_{t_q}^{t_{q+1}} \left( a_i \left( 
		\frac{\int_{t_q}^{t_{q+1}}b_i(t)dt}{t_{q+1} - t_q}\right) +c_i\right)dt\\
	 	& = \int_{t_q}^{t_{q+1}} a_i \left( \frac{\int_{t_q}^{t_{q+1}}b_i(t)dt}
		{t_{q+1} - t_q}\right) + c_i(t_{q+1} - t_q)\\
	 	& = a_i \int_{t_q}^{t_{q+1}}b_i(t)dt +c_i(t_{q+1} - t_q)\\
	 	& = \int_{t_q}^{t_{q+1}} f_i(b_i(t))dt
	\end{array}
	$

	Donc, nous pouvons poser $\forall t \in \inter{t_q}{t_{q+1}}, b_i(t)=
	\frac{\int_{t_q}^{t_{q+1}}b_i(t)dt}{t_{q+1} - t_q}$ et ceci ne change pas 
	la faisabilité de la solution.\\
\end{proof}

Intuitivement, on pourrait penser que, de plus, il n'est pas nécéssaire de 
changer une fonction $b_i(t)$ à une date différente de $r_i$ ou de $ft_j$. Or, 
ce n'est pas le cas, comme le montre le contre-exemple suivant:

\paragraph{Contre-exemple}
	Soit l'intance suivante de CECSP:\\
	\begin{center}
	\begin{tabular}{|c|cccccc|}
	\hline
	$i$&$r_i$&$d_i$&$b_i^{min}$&$b_i^{max}$&$W_i$&$f_i(b)$\\
	\hline
	1 & 0 & 3 & 1 & 3 & 5 & $b$\\
	2 & 1 & 6 & 1 & 2 & 8 & $b$\\
	3 & 2 & 8 & 1 & 1 & 4 & $b$\\
	4 & 2 & 8 & 1 & 2 & 7 & $b$\\
	\hline
	\end{tabular}
	\end{center}
	dont un ordonnancement possible avec une ressource de capacité $B=3$ serait le suivant : \\
\begin{figure}[ht]
	\begin{tikzpicture}
	[inner sep = 0.0pt]
		\node (O) at (0,0) {};
		\node (hg) at (0,3) {};
		\node (hd) at (8,3) {};
		\node (bd) at (8,0) {};

		\path[draw=black,fill=red!50] (O.center) -- (1,0) -- (1,2) -- (3,2) -- (3,3) -- (0,3) -- cycle;
		\path[draw=black,fill=green!50] (1,0) -- (4,0) -- (4,1) -- (6,1) -- (6,2) -- (1,2) -- cycle;
		\path[draw=black,fill=blue!50] (4,0) rectangle (8,1);
		\path[draw=black,fill=orange!50] (6,1) -- (8,1) -- (8,3) -- (3,3) -- (3,2) -- (6,2) -- cycle;
		
		\draw[dashed] (0,0) node[below=0.2cm] {$r_1$} -- (0,3.5) node[above] {$0$};
		\draw[dashed] (1,0) node[below=0.2cm] {$r_2$}-- (1,3.5) node[above] {$1$};
		\draw[dashed] (2,0) node[below=0.2cm] {$r_3$} node[below=0.5cm] {$r_4$} -- (2,3.5) node[above] {$2$};
		\draw[dashed]  (3,0) node[below=0.2cm] {$d_1$} -- (3,3.5) node[above] {$3$};
		\draw[dashed]  (6,0) node[below=0.2cm] {$d_2$} -- (6,3.5) node[above] {$6$};
		\draw[dashed]  (8,0) node[below=0.2cm] {$d_3$} node[below=0.5cm] {$d_4$}-- (8,3.5) node[above] {$8$};
				
		\draw (1,2) node[above right=0.3cm] {\Large \color{DarkRed} $1$};
		\draw (2,1) node[right=0.4cm] {\Large \color{DarkGreen} $2$};
		\draw (6,0) node[above left=0.3cm] {\Large \color{DarkBlue} $3$};
		\draw (6,2) node[above left=0.3cm] {\Large \color{OrangeRed} $4$};
		
		\draw[->] (-1,0) -- (9,0);
		\draw (O.center) -- (hg.center) -- (hd.center) -- (bd.center);
	\end{tikzpicture}
	\caption{Ordonnancement pour une ressource de capacité 3}
	\label{CErifti}
\end{figure}

Si on applique le raisonnement énergétique dans l'intervalle $\inter{3}{6}$, on remarque que, pour consommer le minimum de ressource, la tâche $2$ doit être calée à gauche et les tâches $3$ et $4$ doivent être calées à droite. Dans cette configuration, la consommation minimale de la ressource et la capacité de l'intervalle sont égales à 9. Donc, dans toute solution, la tâche $2$ est calée à gauche et les tâches $3$ et $4$ sont calées à droite. De plus, dans l'intervalle $\inter{3}{6}$, la seule configuration possible est celle de la figure~\ref{CErifti} donc la tâche $3$ commence forcément à $t=4$ et $\forall i \in A,\ t\neq ft_j$ et $t\neq r_3$
\end{document}
